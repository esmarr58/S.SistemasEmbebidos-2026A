\documentclass[12pt,letterpaper]{report}

% ==================================================
% IDIOMA Y FORMATO
% ==================================================
\usepackage[spanish,es-nodecimaldot]{babel}
\usepackage[utf8]{inputenc}
\usepackage[T1]{fontenc}
\usepackage{lmodern}
\usepackage{geometry}
\geometry{margin=2.5cm}
\usepackage{setspace}
\onehalfspacing

% ==================================================
% IMÁGENES Y TABLAS
% ==================================================
\usepackage{graphicx}
\usepackage{float}
\usepackage{booktabs}
\usepackage{tabularx}
\usepackage{hyperref}
\usepackage{xcolor}

% ==================================================
% ENCABEZADO FORMAL
% ==================================================
\usepackage{fancyhdr}
\pagestyle{fancy}
\fancyhf{}
\fancyhead[L]{CUCEI -- Universidad de Guadalajara}
\fancyhead[R]{Ciclo 2026A}
\fancyfoot[C]{\thepage}

% ==================================================
\begin{document}

% ==================================================
% PORTADA
% ==================================================
\begin{titlepage}
\centering

\includegraphics[width=4cm]{Reportes/Imagenes/udg.png}\\[1cm]

{\large \textbf{UNIVERSIDAD DE GUADALAJARA}}\\
{\large Centro Universitario de Ciencias Exactas e Ingenierías (CUCEI)}\\
{\large Departamento de Electrónica}\\[1.2cm]

{\Large \textbf{Seminario de Programación de Sistemas Embebidos}}\\
{\large Ciclo Escolar 2026A}\\[1.5cm]

{\Huge \textbf{Reporte Técnico de Práctica \#1}}\\
{\Large GPIO: Salidas Digitales y Display de 7 Segmentos}\\[1.5cm]

\begin{tabular}{rl}
\textbf{Alumno:} & \rule{9cm}{0.4pt} \\
\textbf{Código:} & \rule{9cm}{0.4pt} \\
\textbf{Carrera:} & \rule{9cm}{0.4pt} \\
\textbf{Grupo:} & \rule{9cm}{0.4pt} \\
\textbf{Profesor:} & Dr. Rubén Estrada Marmolejo \\
\textbf{Fecha de entrega:} & \rule{9cm}{0.4pt} \\
\end{tabular}

\vfill
Guadalajara, Jalisco, México\\
2026

\end{titlepage}

% ==================================================
\chapter*{Resumen}
\addcontentsline{toc}{chapter}{Resumen}

\textbf{Qué debe incluir:}
Contexto, objetivo, modificación realizada, resultado y conclusión general.
Extensión: 150–200 palabras.
Redactar en tercera persona.

\textbf{Ejemplo incorrecto:}
\textcolor{red}{“En esta práctica hicimos el display y al final sí funcionó.”}

\textbf{Ejemplo correcto:}
\textcolor{blue}{“En la presente práctica se implementó el control de un display de 7 segmentos mediante salidas GPIO de la ESP32-S3. Se verificó el funcionamiento del código base y posteriormente se modificó la secuencia de salida. El sistema operó de manera estable y sin errores intermitentes.”}

\vspace{4cm}

\textbf{Palabras clave:} GPIO, ESP32-S3, display de 7 segmentos, lógica digital.

\clearpage
\tableofcontents
\clearpage

% ==================================================
\chapter{Introducción}

\textbf{Qué debe incluir:}
Explicación técnica del contexto.
No describir lo que hizo el alumno paso a paso.

\textbf{Incorrecto:}
\textcolor{red}{“Primero conectamos los cables y luego subimos el código.”}

\textbf{Correcto:}
\textcolor{blue}{“Los pines GPIO permiten controlar dispositivos externos mediante señales digitales. El display de 7 segmentos es un dispositivo ampliamente utilizado para representar información numérica en sistemas embebidos.”}

\vspace{4cm}

% ==================================================
\chapter{Objetivos}

\section{Objetivo general}

Debe redactarse en infinitivo.

\textbf{Incorrecto:}
\textcolor{red}{“El objetivo es que funcione.”}

\textbf{Correcto:}
\textcolor{blue}{“Implementar y validar el control de un display de 7 segmentos mediante salidas GPIO.”}

\vspace{2cm}

\section{Objetivos específicos}

\begin{itemize}
    \item Verificar el funcionamiento del código base.
    \item Modificar únicamente la secuencia de salida.
    \item Validar el comportamiento en hardware físico.
\end{itemize}

% ==================================================
\chapter{Materiales y Metodología}

\section{Materiales utilizados}

Listar únicamente componentes reales utilizados.

\textbf{Incorrecto:}
\textcolor{red}{“Todo lo del laboratorio.”}

\textbf{Correcto:}
\textcolor{blue}{“ESP32-S3, display de 7 segmentos tipo cátodo común, siete resistencias de 220Ω, protoboard y cables de conexión.”}

\vspace{2cm}

\section{Metodología}

Describir procedimiento técnico ordenado.

\textbf{Incorrecto:}
\textcolor{red}{“Movimos cosas del código hasta que funcionó.”}

\textbf{Correcto:}
\textcolor{blue}{“Se ejecutó inicialmente el código base sin modificaciones para verificar el conteo del 0 al 9. Posteriormente, se modificó exclusivamente la estructura que define la secuencia de salida.”}

\vspace{4cm}

% ==================================================
\chapter{Resultados}

\section{Secuencia implementada}

\begin{center}
\begin{tabular}{cccccccc}
\toprule
Paso & 1 & 2 & 3 & 4 & 5 & 6 & 7 \\
\midrule
Valor mostrado & & & & & & & \\
\bottomrule
\end{tabular}
\end{center}

\section{Validación experimental}

Describir lo observado de manera objetiva.

\textbf{Incorrecto:}
\textcolor{red}{“Creo que funcionó bien.”}

\textbf{Correcto:}
\textcolor{blue}{“La secuencia implementada se visualizó correctamente en el display. No se observaron reinicios ni fallas intermitentes.”}

\vspace{4cm}

% ==================================================
\chapter{Discusión}

Analizar dificultades y soluciones técnicas.

\textbf{Incorrecto:}
\textcolor{red}{“Fue difícil porque no sabía qué hacer.”}

\textbf{Correcto:}
\textcolor{blue}{“Una dificultad relevante fue identificar el tipo de display. Esta situación se resolvió mediante la verificación del comportamiento lógico de los segmentos y la comparación con la hoja de datos.”}

\vspace{4cm}

% ==================================================
\chapter{Conclusiones}

Responder: ¿Qué se aprendió técnicamente?

\textbf{Incorrecto:}
\textcolor{red}{“Aprendí mucho.”}

\textbf{Correcto:}
\textcolor{blue}{“Se comprendió el funcionamiento de las salidas digitales GPIO y la importancia de validar modificaciones de software mediante pruebas experimentales controladas.”}

\vspace{3cm}

% ==================================================
\chapter{Cuestionario}

Copiar 5 preguntas del banco y responderlas con claridad técnica.
Cada respuesta debe tener mínimo 2–4 líneas.

\vspace{4cm}

% ==================================================
\end{document}
